\documentclass[11pt,a4paper]{article}
\usepackage[T1]{fontenc}
\usepackage{isabelle,isabellesym}

% this should be the last package used
\usepackage{pdfsetup}

% urls in roman style, theory text in math-similar italics
\urlstyle{rm}
\isabellestyle{it}


\begin{document}

\title{Power Operator for Lists}
\author{\v{S}t\v{e}pán Holub, Martin Ra\v{s}ka, \v{S}t\v{e}pán Starosta and Tobias Nipkow}
\maketitle

\begin{abstract}
  This entry defines the power operator \verb!xs ^^ n!, the \texttt{n}-fold
  concatenation of \texttt{xs} with itself.

  Much of the theory is taken from the AFP entry
  \href{https://www.isa-afp.org/entries/Combinatorics_Words.html}
  {Combinatorics on Words Basics} where
  the operator is called \verb!^@!. This new entry uses the
  standard overloaded \verb!^^! syntax and is aimed at becoming the
  central theory of the power operator for lists that can be extended easily.
\end{abstract}

% include generated text of all theories
\input{session}

\end{document}
