\documentclass[11pt,a4paper]{article}
\usepackage[T1]{fontenc}
\usepackage{isabelle,isabellesym}

% further packages required for unusual symbols (see also
% isabellesym.sty), use only when needed

\usepackage{amssymb}
  %for \<leadsto>, \<box>, \<diamond>, \<sqsupset>, \<mho>, \<Join>,
  %\<lhd>, \<lesssim>, \<greatersim>, \<lessapprox>, \<greaterapprox>,
  %\<triangleq>, \<yen>, \<lozenge>

\usepackage{latexsym}
\usepackage{wasysym}

%\usepackage{eurosym}
  %for \<euro>

\usepackage[only,bigsqcap,bigparallel,fatsemi,interleave,sslash]{stmaryrd}
  %for \<Sqinter>, \<Parallel>, \<Zsemi>, \<Parallel>, \<sslash>

%\usepackage{eufrak}
  %for \<AA> ... \<ZZ>, \<aa> ... \<zz> (also included in amssymb)

%\usepackage{textcomp}
  %for \<onequarter>, \<onehalf>, \<threequarters>, \<degree>, \<cent>,
  %\<currency>

% this should be the last package used
\usepackage{pdfsetup}

% urls in roman style, theory text in math-similar italics
\urlstyle{rm}
\isabellestyle{it}

% for uniform font size
%\renewcommand{\isastyle}{\isastyleminor}


\begin{document}

\title{Notes on Gödel's and Scott's Variants of the Ontological
  Argument (Isabelle/HOL dataset)}
\author{Christoph Benzm{\"u}ller and Dana S. Scott}
\maketitle

\begin{abstract}
Experimental studies with Isabelle/HOL on Kurt G\"odel's modal
ontological argument and Dana Scott's variant of it are presented.
They implicitly answer some questions that the authors have received
over the last decade(s). In addition, some new results are reported.

Our contribution is explained in full detail in \cite{J75}.  This document
presents the corresponding Isabelle/HOL dataset (which is only slightly modified
to meet AFP requirements).
\end{abstract}

\tableofcontents

% sane default for proof documents
\parindent 0pt\parskip 0.5ex


\section{Introduction}
The Isabelle/HOL dataset associated with 
\cite{J75} is presented. Compared to previous work on G\"odel's modal
ontological argument as published in the Archive of Formal
Proofs (AFP) \cite{SimplifiedOntologicalArgument-AFP,Types_Tableaus_and_Goedels_God-AFP,GoedelGod-AFP},
our dataset addresses several relevant and in some cases
novel aspects, which are combined here in a single publication,
including:
\begin{enumerate}
  \item For the first time, G\"odel's original manuscript
    \cite{GoedelNotes} has been formalized as closely as possible.
   \item The inconsistency of the postulates in G\"odel's original manuscript is explained in detail.
   \item Two different ways of eliminating this inconsistency are presented, one of which is novel.
   \item Scott's variant \cite{ScottNotes} of G\"odel's original
     manuscript is presented and compared with G\"odel's original
     variant.
  \item In addition to logics S5 and K, the above
    variants are also studied for logic S4.
  \item The above variants are tested for various combinations of
    of possibilist and actualist quantifiers for individuals (which has not been done systematically in previous publications).
  \item Concepts of evil are examined and the derivability of evil is critically questioned.
\end{enumerate}  
The purpose of this AFP publication is essentially twofold. One
motivation is to make the data sources associated with \cite{J75}
available in a sustainable, well-maintained way. The other motivation is to support
university education in higher-order modal logic by 
providing a small dataset for reuse that illustrates a systematically explored
philosophical argument, emphasizing in particular 
different notions of quantification.

Compared to \cite{J75}, the Isabelle sources presented here have been slightly modified to meet some AFP requirements. This concerns the commenting out of calls to sledghemmer (to reduce computational resources) and 
some minor reformatting (e.g. insertion of new lines). The formalization code itself remains unchanged.

% generated text of all theories
\input{session}

% optional bibliography
\bibliographystyle{abbrv}
\bibliography{root}

\end{document}

%%% Local Variables:
%%% mode: latex
%%% TeX-master: t
%%% End:
